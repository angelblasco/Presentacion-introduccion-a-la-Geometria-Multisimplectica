\documentclass{beamer}
\usepackage[utf8]{inputenc}
\usepackage{amsmath, amssymb}
\usepackage[spanish]{babel}
\usepackage{graphicx}
\usepackage{biblatex} %Imports biblatex package
\addbibresource{referencias.bib} %Import the bibliography file
\usepackage{physics}
\usepackage{hyperref}
\usetheme{Madrid}

\title{Geometría multisimpléctica: primeras definiciones y propiedades básicas.}
\author{Ángel Blasco Muñoz}
\institute{Escuela Internacional de Doctorado, Universidad Nacional de Educación a Distancia (UNED), España.}
\date{8 de enero de 2026}
% Logo only on title page
\titlegraphic{
    \includegraphics[width=3cm]{images/uned_logo_eiduned.png}
}
%The next block of commands puts the table of contents at the 
%beginning of each section and highlights the current section:

\AtBeginSection[]
{
  \begin{frame}
    \frametitle{Contenidos.}
    \tableofcontents[currentsection]
  \end{frame}
}
\begin{document}

%----------------------------------------
\begin{frame}
  \titlepage
\end{frame}
%---------------------------------------------------------

\begin{frame}{Referencias.}
\printbibliography %Prints bibliography
\end{frame}
%----------------------------------------
\begin{frame}
\frametitle{Contenidos.}
\tableofcontents
\end{frame}
%----------------------------------------


\section{Introducción.}

%---------------------------------------
\begin{frame}{Introducción.}
Se pretende ofrecer una introducción a la Geometría Multisimpléctica basada en el estudio de \cite{cantrijn_hamiltonian_1996} y \cite{cantrijn_geometry_1999} y ver sus propiedades básicas, las cuales forman un marco abstracto de estudio para la teoría clásica de campos, la geometría mecánica, la dinámica... y demás ramas relacionadas con la física-matemática.
\end{frame}

%----------------------------------------

\section{Espacios vectoriales multisimplécticos.}

%---------------------------------------------------------
\begin{frame}{Espacios vectoriales multisimplécticos.}
\begin{block}{Definición}
Un espacio vectorial multisimpléctico de orden $k+1$ es un par $(\mathcal{V}, \omega)$ formado por un espacio vectorial $\mathcal{V}$ y una $(k+1)$-forma en $\mathcal{V}$ no degenerada.
\end{block}
\pause
$\omega$ se denomina \textit{forma multisimpléctica} (de grado $k+1$). 
\pause
$i_{v} \omega = 0 \iff v=0$ para $v \in \mathcal{V}$.\\
\pause
La no-degeneración de la forma multisimpléctica significa que la aplicación inducida:
\[\hat{\omega}: \mathcal{V} \rightarrow \land^{k}\mathcal{V^{*}}
\]
\[
v \mapsto i_{v}\omega
\]
es inyectiva.
\end{frame}

%----------------------------------------
\begin{frame}{Espacios vectoriales multisimplécticos.}
Dos espacios vectoriales multisimplécticos $(\mathcal{V}, \omega)$ y $(\tilde{\mathcal{V}}, \tilde{\omega})$ del mismo orden $(k+1)$, son isomorfos si existe un isomorfismo
\[
\Psi : \mathcal{V} \rightarrow \tilde{\mathcal{V}}
\]
tal que:
\[
\tilde{\omega}(\Psi(v_{1}), \ldots ,\Psi(v_{k+1}))=\omega (v_{1}, \ldots , v_{k+1})
\]
para todo $v_{i} \in \mathcal{V}, \ (i=1, \ldots , k+1)$.
\end{frame}
%----------------------------------------
\begin{frame}{Espacios vectoriales multisimplécticos.}
Por \cite{abraham_foundations_2008} sabemos que cualquier espacio vectorial $V$, el espacio $V \times V^{*}$ admite una forma simpléctica canónica $\Omega$ definida:
\[
\Omega((v_{1},\alpha_{1}),(v_{2},\alpha_{2}))=\alpha_{2}(v_{1})-\alpha_{1}(v_{2})
\]
\pause
\textit{¿Esta estructura tiene su extensión natural en el marco multisimpléctico?}
\pause
Si. Dado un $k$ con $1 \leq k \leq$ dim $V$, el espacio $V \times \land^{k}V^{*}$ puede ser equipado con una $(k+1)$-forma canónica $\Omega$ definida:
\[
\Omega((v_{1},\alpha_{1}),\ldots ,(v_{k+1},\alpha_{k+1}))=
\sum_{i=1}^{k+1}(-1)^{i}\alpha_{i}(v_{1}, \ldots , \hat{v_{i}}, \ldots , v_{k+1})
\]
\pause
\begin{alertblock}{Proposición:}
$(V \times \land^{k}V^{*},\Omega)$ es un espacio vectorial multisimpléctico de orden $k+1$.
\end{alertblock}
\end{frame}
%----------------------------------------
\begin{frame}{Espacios vectoriales multisimplécticos.}
En adelante, por simplicidad, utilizaremos la notación $\mathcal{V}^{k}_{V}=V \times \wedge^{k}V^{*}$.

\end{frame}
%----------------------------------------
\begin{frame}{Espacios vectoriales multisimplécticos.}
Consideramos la siguiente aplicación lineal sobreyectiva
\[
\pi : V \longrightarrow W
\]
y su secuencia exacta corta
\[
0 \rightarrow \text{ker}\pi \hookrightarrow V \xrightarrow{\pi} W \rightarrow 0
\]
\pause
Denotamos por $\wedge^{k}_{r} \pi$ el espacio de las $k$-formas exteriores.
\pause
\[
\alpha \in \wedge^{k}_{r} \pi \Longleftrightarrow i_{v_{1}\wedge \ldots \wedge v_{r+1}}\alpha =0
\] 
\end{frame}
%----------------------------------------
\begin{frame}{Espacios vectoriales multisimplécticos.}
\[
\wedge^{k}_{0} \pi \subseteq \wedge^{k}_{1} \pi \subseteq \ldots \subseteq \wedge^{k}_{k-1} \pi \subseteq \wedge^{k} V^{*}
\]
Denotamos
\[
\mathcal{V}^{(k,r)}_{\pi} = V \times \wedge^{k}_{r} \pi
\]
\pause
Claramente, para $0 \leq r < k$ tenemos que $\mathcal{V}^{(k,r)}_{\pi}$ es un subespacio de $\mathcal{V}^{k}_{V}$\\
-CONTAR pg 308 de \cite{cantrijn_geometry_1999}
\pause
\begin{alertblock}{Proposición:}
Para cada $r$, con $0 \leq r \leq k-1$ y $k-r \leq \text{dim}W$, $(\mathcal{V}^{(k,r)}_{\pi}, \Omega)$ es un espacio vectorial multisimpléctico de orden $k+1$.
\end{alertblock}
\end{frame}
%----------------------------------------
\begin{frame}{Espacios vectoriales multisimplécticos.}
La anterior proposición nos indica que para cualquier fibrado $\pi$ de un espacio vectorial $V$, hay una familia de subespacios vectoriales multisimplécticos de $(\mathcal{V}^{k}_{V}, \Omega)$.\\
La estructura multisimpléctica obtenida al restringir $\Omega$ a $\mathcal{V}^{(k,l)}_{\pi}= V \times \wedge^{k}_{l} \pi$ es el modelo lineal para la estructura multisimpléctica canónica que surge en la formulación Hamiltoniana covariante de la teoría de campos (Ver \cite{carinena_multisymplectic_1991}).
\end{frame}
%----------------------------------------
\section{Subespacios característicos.}

%---------------------------------------------------------
\begin{frame}{Subespacios característicos.}
Sea $(\mathcal{V}, \omega)$ un espacio vectorial multisimpléctico con una $(k+1)$-forma no degenerada $\omega$, y sea $W$ un subespacio de $\mathcal{V}$. Sea $l$, con $1 \leq l \leq k$,
\begin{block}{Definición}
El $l$-ésimo complemento ortogonal de $W$ es un subespacio lineal de $\mathcal{V}$ definido como:
\[
W^{\perp , l} = \left\lbrace v \in \mathcal{V} \:| \:
i_{v \wedge w_{1} \wedge \ldots \wedge w_{l}}\omega = 0 \text{ para todo }w_{i} \in W, i=1, \ldots , l \right\rbrace 
\]
\end{block}
\pause
\[
W^{\perp , 1} \subseteq W^{\perp , 2} \subseteq \ldots \subseteq W^{\perp , k}
\]
\pause
\[
W^{\perp , l} = \mathcal{V} \text{ siempre que } l > \text{dim } W
\]
\end{frame}
%----------------------------------------
\begin{frame}{Subespacios característicos.}
En \cite{cantrijn_geometry_1999} se enuncian y demuestran las siguientes propiedades:
\begin{itemize}
	\item $\left\lbrace 0 \right\rbrace^{\perp ,l}= \mathcal{V} \text{ y } \mathcal{V}^{\perp , l}= \left\lbrace 0 \right\rbrace $;
	\item $U \subset W \Longrightarrow W^{\perp , l} \subset U^{\perp , l}$;
	\item $(U+W)^{\perp , l} \subset W^{\perp , l} \cap U^{\perp , l}$;
	\item $U^{\perp , l_{1}} \cap W^{\perp , l_{2}} \subset (U+W)^{\perp, l_{1}+l_{2}-1} \text{ para } l_{1}+ l_{2} \leq k+1$;
	\item $U^{\perp , l_{1}} + W^{\perp , l_{2}} \subset (U \cap W)^{\perp, \overline{l}} \text{ para } \overline{l}=\text{max}\left\lbrace l_{1},l_{2} \right\rbrace $;
\end{itemize}
\end{frame}
%---------------------------------------
\begin{frame}{Subespacios característicos.}
$W$ es un subespacio de un espacio vectorial multisimpléctico $(\mathcal{V}, \omega)$ de orden $k+1$, entonces:
\begin{itemize}
\item l-isotrópico si $W \subset W^{\perp ,l}$
\item l-coisotrópico si $W^{\perp ,l} \subset W$
\item l-Lagrangiano si $W = W^{\perp ,l}$
\end{itemize}

\end{frame}
%---------------------------------------
\section{Caracterización de las estructuras multisimplécticas canónicas.}
%----------------------------------------
\begin{frame}{Caracterización de las estructuras multisimplécticas canónicas.}
Por \cite{libermann_symplectic_1987} sabemos que:\\
$(\mathcal{V}, \omega)$ sea un espacio vectorial simpléctico y $L$ un subespacio Lagrangiano de $\mathcal{V}$ (que siempre existe)
\pause
\[
(\mathcal{V}, \omega) \cong (\mathcal{V}^{1}_{L}, \Omega)
\]
con
\[
\Omega((v_{1},\alpha_{1}),(v_{2},\alpha_{2}))=\alpha_{2}(v_{1})-\alpha_{1}(v_{2})
\]
\pause
Todos los espacios vectoriales simplécticos de la misma dimensión "son parecidos", en el sentido de que son isomorfos a algún $(\mathcal{V}^{1}_{L}, \Omega)$.
\end{frame}
%---------------------------------------
\begin{frame}{Caracterización de las estructuras multisimplécticas canónicas.}
¿Y en el marco multisimpléctico?\\
\pause
¿Como tienen que ser los espacios vectoriales multisimplécticos para que sean isomorfos a alguno del tipo $(\mathcal{V}^{k}_{V}, \Omega)$?
\end{frame}
%----------------------------------------
\section{Variedades multisimplécticas.}
%----------------------------------------
\begin{frame}{Variedades multisimplécticas.}
  
\end{frame}
%------------------------------------------
\section{Estructuras Hamiltonianas en variedades multisimplécticas.}
%----------------------------------------
\begin{frame}{Estructuras Hamiltonianas en variedades multisimplécticas.}
  
\end{frame}

%----------------------------------------
\section{Modelos canónicos para variedades multisimplécticas. Coordenadas de Darboux.}
%-----------------------------------------
\begin{frame}{Modelos canónicos para variedades multisimplécticas. Coordenadas de Darboux.}

\end{frame}

%----------------------------------------

\end{document}